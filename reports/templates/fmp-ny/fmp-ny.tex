\documentclass[]{tufte-handout}

% ams
\usepackage{amssymb,amsmath}

\usepackage{ifxetex,ifluatex}
\usepackage{fixltx2e} % provides \textsubscript
\ifnum 0\ifxetex 1\fi\ifluatex 1\fi=0 % if pdftex
  \usepackage[T1]{fontenc}
  \usepackage[utf8]{inputenc}
\else % if luatex or xelatex
  \makeatletter
  \@ifpackageloaded{fontspec}{}{\usepackage{fontspec}}
  \makeatother
  \defaultfontfeatures{Ligatures=TeX,Scale=MatchLowercase}
  \makeatletter
  \@ifpackageloaded{soul}{
     \renewcommand\allcapsspacing[1]{{\addfontfeature{LetterSpace=15}#1}}
     \renewcommand\smallcapsspacing[1]{{\addfontfeature{LetterSpace=10}#1}}
   }{}
  \makeatother

\fi

% graphix
\usepackage{graphicx}
\setkeys{Gin}{width=\linewidth,totalheight=\textheight,keepaspectratio}

% booktabs
\usepackage{booktabs}

% url
\usepackage{url}

% hyperref
\usepackage{hyperref}

% units.
\usepackage{units}


\setcounter{secnumdepth}{-1}

% citations

% pandoc syntax highlighting

% longtable
\usepackage{longtable,booktabs}

% multiplecol
\usepackage{multicol}

% strikeout
\usepackage[normalem]{ulem}

% morefloats
\usepackage{morefloats}


% tightlist macro required by pandoc >= 1.14
\providecommand{\tightlist}{%
  \setlength{\itemsep}{0pt}\setlength{\parskip}{0pt}}

% title / author / date
\date{}


\begin{document}





\pagenumbering{gobble} \vspace*{30pt}

\huge FOREST MANAGEMENT PLAN

\vspace{30pt}

\LARGE McGann Property

\vspace{22pt}

\normalsize Town of Altona, Clinton County, New York

\vspace{70pt}

\small 220.2 acres

\vspace{18pt}

Parcel IDs:

\vspace{8pt}

NA

\vspace{10pt}

Deed (Liber/Pages): NA

\vspace{10pt}

480-a certification number: 09-047

\vspace{10pt}

Original certication date: NA

\vspace{60pt}

Prepared by:

\includegraphics[width=0.5\linewidth,height=0.5\textheight]{C:/Users/Neal/projects/cruise/logo-name-small}

John D. Foppert \& Neal F. Maker\\
96 Durocher Road\\
Saranac, NY 12981

\vspace{30pt}

\large July 10, 2020

\pagebreak
\pagenumbering{arabic}

\section{Introduction}\label{introduction}

This plan covers the fifteen year period from 2019 to 2034. It lays out
the near- and medium-term actions that should guide the development of
the McGann forest. It also qualifies portions of the property for
continued enrollment in the 480-a Forest Tax Law program and
commensurate reduction in property taxes.\footnote{Further information
  about 480-a can be found at the New York Department of Conservation's
  website: \url{https://www.dec.ny.gov/lands/5236.html}. \vspace{20pt}}
Owners participating in the program are obliged to manage enrolled
portions of their property according to their approved forest management
plan and to make any reasonable investments for improvement that the
plan recommends. Its recommendations were developed in accordance with
the principles and practices of scientifically sound forestry, as
described in the relevant management guidelines, textbooks and academic
journals.

\section{Property Description}\label{property-description}

The McGann forest represents a well-managed resource positioned to
continue generating value on a sustainable basis. NA percent of the
220.2 acre property is productive forestland to be managed according to
this plan. The property is located in the Adirondack Park in the Town of
Altona, Clinton County, New York. The property is accessed via the
Jerusalem Road, a former town road that was formally abandoned. The road
is passable but has not been well maintained. Internal access to the
property is provided by an extensive network of skid trails established
during recent harvesting operations. Property-wide, elevations range
from 1250 to 1460 feet above mean sea level. The primary water resources
on the property are an unnamed stream the flows to the north and roughly
bisects the property. This stream has its headwaters in a neighboring
lot, roughly half a mile to the south. A few small, intermittent
tributaries originate on the property and join the stream as it follows
a shallow, curving channel before crossing the boundary line to the
north. From there it meanders another three miles until it enters the
Great Chazy River in Alder Bend. There is also a smaller, intermittent
stream in the eastern side of the property. This stream joins the larger
stream off the property, about halfway along its course to the Great
Chazy. The Great Chazy River flows northeast, entering Lake Champlain
near Rouses Point, eventually reaching the sea via the Richelieu and
St.~Lawrence Rivers. Three areas of forested wetlands are present in the
eastern half of the stand, covering an area of less than five acres in
aggregate. The property's boundaries are well established and clearly
marked. The southern boundary is defined by Jerusalem Road. The property
abuts state land to the west, with the boundary very clearly blazed with
yellow paint and posted with Forest Preserve signs. To the north, the
property abuts a separate lot also owned by the McGanns but cleared
delineated with red painted blazes. The property abuts one private
parcel to the east and the shared boundary is discernable and marked
with red paint. Soils, forest health, and other pertinent topics are
discussed in the individual stand area descriptions that follow.

\section{Principles, Goals \& Strategies For Forest
Management}\label{principles-goals-strategies-for-forest-management}

\subsection{Conservation}\label{conservation}

The ecological functioning, productive capacity and biological diversity
of the forest resource should be maintained or improved over time so as
to provide opportunities for the current or future landowners to
continue to enjoy and use the property. A management strategy that is
sustainable in the long-term and viable in the short- and medium-terms
offers a strong measure of protection against future development or
conversion.

\subsection{Timber management}\label{timber-management}

Management should provide regular returns from timber harvesting.
Long-term value growth is provided by maintaining full site occupancy
with healthy trees capable of producing high quality sawtimber or
veneer. Tree species which yield sought-after, high-value wood should be
promoted within each stand or, when regenerating a new stand, attention
should be paid to creating stand conditions that favor the establishment
of those species. At a property-wide scale, a variety of species should
be maintained, providing options for seizing future market opportunities
and a hedge against species-specific market depreciation. Among desired
species, additional preference should be given to individual trees of
sufficient vigor and grade-potential for strong future value growth.
Consideration of economic efficiency should inform the timing and
coordination of infrastructure investments and stand maintenance,
improvement and harvest operations.

\section{Stand Descriptions \& Management
Recommendations}\label{stand-descriptions-management-recommendations}

Presented below are detailed stand-by-stand descriptions of the forest,
the long-term structural, compositional and functional goals for each
stand, and the near-term silvicultural treatments or management
activities that have been prescribed to advance each stand toward those
goals. The data presented in the following pages was obtained from a
thorough inventory of the property in November of 2019. General
conditions were assessed qualitatively in conjunction with quantitative
sampling. Observational notes and sample summary statistics together
provide the basis for the stand descriptions and management
recommendations. All sampling was done using a systematic sample and
variable radius plots. In stands with uneven-aged structures, all trees
6" in diameter at breast height (dbh) and larger were measured in each
plot. In stands with even-aged structures, all main-canopy trees were
measured in each plot.

When contractors are used to implement silvicultural prescriptions, they
should be highly skilled, properly equipped, fully insured, and closely
supervised. A professional forester should prepare and administer
commercial treatments, and logging operations should be timed to
coincide with favorable weather conditions (working on wet soils only
when they are frozen, for instance) and favorable timber markets. The
dates assigned to timber harvests and other management activities
prescribed in this plan are intended to guide, rather than constrain,
forest management. To accommodate dynamic markets and variable weather,
scheduled timber harvests may be advanced or delayed by one year from
the date indentified in this plan; if operational or economic conditions
change substantially, the management schedule may be further revised by
an ammendment to this plan.

The property should be reassessed in 2024 and the findings brought to
bear on a reassessment of the goals and strategies proposed in this
plan, leading to a formal management plan update.

\newpage

\section{Management Schedule}\label{management-schedule}

\small 2020 - Full management plan revision; boundary line maintenance;
Stand 1: Group selection harvest; Stand 2: Group selection harvest

\vspace{5pt}

\noindent 2021 - No scheduled activity

\vspace{5pt}

\noindent 2022 - No scheduled activity

\vspace{5pt}

\noindent 2023 - No scheduled activity

\vspace{5pt}

\noindent 2024 - No scheduled activity

\vspace{5pt}

\noindent 2025 - Management plan update; boundary line maintenance

\vspace{5pt}

\noindent 2026 - No scheduled activity

\vspace{5pt}

\noindent 2027 - No scheduled activity

\vspace{5pt}

\noindent 2028 - No scheduled activity

\vspace{5pt}

\noindent 2029 - No scheduled activity

\vspace{5pt}

\noindent 2030 - Management plan update; boundary line maintenance

\vspace{5pt}

\noindent 2031 - No scheduled activity

\vspace{5pt}

\noindent 2032 - No scheduled activity

\vspace{5pt}

\noindent 2033 - No scheduled activity

\vspace{5pt}

\noindent 2034 - Full management plan revision; boundary line
maintenance

\newpage

\section{Area 1}\label{area-1}

Northern hardwood\\
\noindent NA acres total\\
\noindent 0.00 acres ineligible wetlands\\
\noindent 0.00 acres ineligible roads and landings

\subsection{Site-specific information}\label{site-specific-information}

\begin{quote}
\begin{itemize}
\tightlist
\item
  \textbf{Soils:}\\
  \indent\indent Becket-Tunbridge-Skerry complex
\end{itemize}
\end{quote}

\begin{quote}
\begin{itemize}
\tightlist
\item
  \textbf{Site Class:}\\
  \vspace{2pt} II (determined from soil mapping and field assessment)
\end{itemize}
\end{quote}

\begin{quote}
\begin{itemize}
\tightlist
\item
  \textbf{Access:}\\
  \vspace{2pt} Frontage on Jerusalem Rd
\end{itemize}
\end{quote}

\begin{quote}
\begin{itemize}
\tightlist
\item
  \textbf{Stand history:}\\
  \vspace{2pt} Stand 1 occupies land that has almost certainly been
  continuously forested since pre-settlement times. The area appears to
  have been burned over in the 1908 fires, which would have been quickly
  followed by establishment of an even-aged hardwood stand. Timber
  harvests were likely conducted on a periodic and largely opportunistic
  basis over the course of the 20th century, resulting in non-uniform
  stocking in the overstory cohort, scattered patches of regeneration,
  and an in increasingly irregular stand structure as the century
  progressed. Heavy harvests in the mid-1990's and again around 2005
  created the residual structure that now defines the stand.
  Demographically, the stand is multi-aged, in that trees from multiple
  age-cohorts are present. Structurally it somewhat resembles a two-aged
  stand, with a distinctive older cohort, originating following various
  mid-century partial harvests, overtopping a younger cohort of
  approximately 25 year old trees. No harvesting or other management
  activities appear to have taken place over the past decade.
\end{itemize}
\end{quote}

\subsection{Current forest
information}\label{current-forest-information}

\begin{quote}
\begin{itemize}
\tightlist
\item
  \textbf{Age Class Structure:}\\
  \vspace{2pt} Uneven-aged\\

  \begin{marginfigure}
  \includegraphics{fmp-ny_files/figure-latex/unnamed-chunk-2-1} \caption[Distributions are approximated with kernel density estimation]{Distributions are approximated with kernel density estimation. Common species are those that account for at least 8 percent of the total stocking and areas under each curve represent species basal areas.}\label{fig:unnamed-chunk-2}
  \end{marginfigure}
\end{itemize}
\end{quote}

\begin{quote}
\begin{itemize}
\tightlist
\item
  \textbf{Species (\% stocking):}\\
  \vspace{2pt} hard maple (44\%), beech (14\%), paper birch (12\%), ash
  (10\%), black cherry (6\%), soft maple (6\%), yellow birch (4\%),
  basswood (2\%)
\end{itemize}
\end{quote}

\begin{quote}
\begin{itemize}
\tightlist
\item
  \textbf{Regeneration:}\\
  \vspace{2pt} Sapling- and pole-sized regeneration is abundant
  throughout the stand. Desirable species such as sugar maple, yellow
  birch and black cherry are well represented among these stems.
\end{itemize}
\end{quote}

\begin{quote}
\begin{itemize}
\tightlist
\item
  \textbf{Forest health:}\\
  \vspace{2pt} Few forest health issues were observed on this portion of
  the property. Beech bark disease is present and affects older beech
  trees. Some maple borer and black knot of cherry damage was also noted
  but does not pose a major threat to long-term stand productivity.
\end{itemize}
\end{quote}

\begin{quote}
\begin{itemize}
\tightlist
\item
  \textbf{Size class structure (\%BA):}\\
  \vspace{2pt} \indent 6-10'': 62\% \textbar{} 11-16'': 34\% \textbar{}
  17-22'': 4\% \textbar{} 23"+: 0\%
\end{itemize}
\end{quote}

\begin{quote}
\begin{itemize}
\tightlist
\item
  \textbf{Standing dead wood (BA by size class):}\\
  \vspace{2pt} \indent \small 6-10``: 1.8 ft\textsuperscript{2}/ac
  \textbar{} 11-16'': 0 ft\textsuperscript{2}/ac \textbar{} 17-22'': 0
  ft\textsuperscript{2}/ac \textbar{} 23+'': 0 ft\textsuperscript{2}/ac
\end{itemize}
\end{quote}

\subsection{Inventory information}\label{inventory-information}

\begin{quote}
\begin{itemize}
\tightlist
\item
  11 points, 10 BAF, November, 2019
\end{itemize}
\end{quote}

\begin{figure}
\includegraphics{fmp-ny_files/figure-latex/unnamed-chunk-5-1} \caption[Points represent individual plots]{Points represent individual plots. Asterisk represnts stand average. Radial lines are quadratic stand diameters.}\label{fig:unnamed-chunk-5}
\end{figure}

\begin{longtable}[]{@{}lrr@{}}
\caption{Measures of stocking for all live trees (total) and acceptable
growing stock.}\tabularnewline
\toprule
& Total & Acceptable\tabularnewline
\midrule
\endfirsthead
\toprule
& Total & Acceptable\tabularnewline
\midrule
\endhead
Basal area (sqft/ac) & 88 & 71\tabularnewline
QSD (in) & 9 & 9\tabularnewline
Stems/ac & 218 & 178\tabularnewline
\bottomrule
\end{longtable}

\subsection{Silvicultural
prescription}\label{silvicultural-prescription}

Groups between 0.25 and 5 acres should be established throughout the
stand; in aggregate, the area of group openings should not exceed 50\%
of the total stand area. Groups should be located so as to: (1) release
patches or contiguous areas of desirable sapling- and pole-sized
regeneration, (2) remove concentrations of unacceptable growing stock or
economically mature stems, and (3) provide a variety of site conditions
and light environments for the establishment of a diverse new
regeneration cohort. Some larger group openings may be structured as
group shelterwoods, with up to 40 sqft. of residual basal area retained
to favor shade-intermediate to shade-tolerant species. Crop trees in the
matrix between group openings should be fully released. Where sufficient
crop tree quality stems are present, selection should achieve a target
density of 70 trees per acre (25 ft. square spacing), though most areas
of the stand are not currently stocked with enough potential crop trees
to achieve this target. Crop tree selection should emphasize on
retention of high-value stems of hard maple, yellow birch and black
cherry; secondarily, high-quality soft maple and paper birch may be
selected. Disease-free beech should also be retained (unless within
around 10-15 feet of a designated crop tree) and can be released or
partially released if competitors are of low- to average-quality.

\newpage

\section{Area 2}\label{area-2}

Northern hardwood\\
\noindent NA acres total\\
\noindent 0.00 acres ineligible wetlands\\
\noindent 0.00 acres ineligible roads and landings

\subsection{Site-specific
information}\label{site-specific-information-1}

\begin{quote}
\begin{itemize}
\tightlist
\item
  \textbf{Soils:}\\
  \indent\indent Becket-Tunbridge-Skerry complex
\end{itemize}
\end{quote}

\begin{quote}
\begin{itemize}
\tightlist
\item
  \textbf{Site Class:}\\
  \vspace{2pt} II (determined from soil mapping and field assessment)
\end{itemize}
\end{quote}

\begin{quote}
\begin{itemize}
\tightlist
\item
  \textbf{Access:}\\
  \vspace{2pt} Frontage on Jerusalem Rd
\end{itemize}
\end{quote}

\begin{quote}
\begin{itemize}
\tightlist
\item
  \textbf{Stand history:}\\
  \vspace{2pt} Stand 2 also appears to have been continuously forested
  since pre-settlement times, though it may not have been affected by
  the 1908 fires. This may account for the historically higher
  proportions of beech in this stand and in the eastern portion of the
  adjacent lot to the north. In any case, a similar pattern of partial
  logging over the course of the 20th century largely obscured the
  possibly distinctive land-use histories of Stands 1 and 2. Unlike
  Stand 1, this stand appears not to have been harvested in the 1990's,
  though the subsequent harvest in the early-2000's seems to have been
  more intensive. This pattern again resulted in an multi-aged stand
  with a `lumpy' age-class distribution and a residual structure
  functionally similar to a two-aged stand. No harvesting or other
  management activities appear to have taken place over the past decade.
\end{itemize}
\end{quote}

\subsection{Current forest
information}\label{current-forest-information-1}

\begin{quote}
\begin{itemize}
\tightlist
\item
  \textbf{Age Class Structure:}\\
  \vspace{2pt} Uneven-aged\\

  \begin{marginfigure}
  \includegraphics{fmp-ny_files/figure-latex/unnamed-chunk-7-1} \caption[Distributions are approximated with kernel density estimation]{Distributions are approximated with kernel density estimation. Common species are those that account for at least 8 percent of the total stocking and areas under each curve represent species basal areas.}\label{fig:unnamed-chunk-7}
  \end{marginfigure}
\end{itemize}
\end{quote}

\begin{quote}
\begin{itemize}
\tightlist
\item
  \textbf{Species (\% stocking):}\\
  \vspace{2pt} hard maple (35\%), beech (21\%), yellow birch (15\%),
  soft maple (11\%), black cherry (5\%), paper birch (4\%), ash (3\%),
  basswood (3\%), other hardwood (2\%), aspen (1\%)
\end{itemize}
\end{quote}

\begin{quote}
\begin{itemize}
\tightlist
\item
  \textbf{Regeneration:}\\
  \vspace{2pt} Sapling- and pole-sized regeneration is abundant
  throughout the stand. Desirable species such as sugar maple, yellow
  birch and black cherry are well represented among these stems.
\end{itemize}
\end{quote}

\begin{quote}
\begin{itemize}
\tightlist
\item
  \textbf{Forest health:}\\
  \vspace{2pt} Few forest health issues were observed on this portion of
  the property. Beech bark disease is present and affects older beech
  trees. Some maple borer and black knot of cherry damage was also noted
  but does not pose a major threat to long-term stand productivity.
\end{itemize}
\end{quote}

\begin{quote}
\begin{itemize}
\tightlist
\item
  \textbf{Size class structure (\%BA):}\\
  \vspace{2pt} \indent 6-10'': 56\% \textbar{} 11-16'': 41\% \textbar{}
  17-22'': 3\% \textbar{} 23"+: 0\%
\end{itemize}
\end{quote}

\begin{quote}
\begin{itemize}
\tightlist
\item
  \textbf{Standing dead wood (BA by size class):}\\
  \vspace{2pt} \indent \small 6-10``: 1.7 ft\textsuperscript{2}/ac
  \textbar{} 11-16'': 2.1 ft\textsuperscript{2}/ac \textbar{} 17-22'': 0
  ft\textsuperscript{2}/ac \textbar{} 23+'': 0 ft\textsuperscript{2}/ac
\end{itemize}
\end{quote}

\subsection{Inventory information}\label{inventory-information-1}

\begin{quote}
\begin{itemize}
\tightlist
\item
  24 points, 10 BAF, November, 2019
\end{itemize}
\end{quote}

\begin{figure}
\includegraphics{fmp-ny_files/figure-latex/unnamed-chunk-10-1} \caption[Points represent individual plots]{Points represent individual plots. Asterisk represnts stand average. Radial lines are quadratic stand diameters.}\label{fig:unnamed-chunk-10}
\end{figure}

\begin{longtable}[]{@{}lrr@{}}
\caption{Measures of stocking for all live trees (total) and acceptable
growing stock.}\tabularnewline
\toprule
& Total & Acceptable\tabularnewline
\midrule
\endfirsthead
\toprule
& Total & Acceptable\tabularnewline
\midrule
\endhead
Basal area (sqft/ac) & 70 & 48\tabularnewline
QSD (in) & 9 & 9\tabularnewline
Stems/ac & 161 & 103\tabularnewline
\bottomrule
\end{longtable}

\subsection{Silvicultural
prescription}\label{silvicultural-prescription-1}

Groups between 0.25 and 5 acres should be established throughout the
stand; in aggregate, the area of group openings should not exceed 50\%
of the total stand area. Groups should be located so as to: (1) release
patches or contiguous areas of desirable sapling- and pole-sized
regeneration, (2) remove concentrations of unacceptable growing stock or
economically mature stems, and (3) provide a variety of site conditions
and light environments for the establishment of a diverse new
regeneration cohort. Some larger group openings may be structured as
group shelterwoods, with up to 40 sqft. of residual basal area retained
to favor shade-intermediate to shade-tolerant species. Crop trees in the
matrix between group openings should be fully released. Where sufficient
crop tree quality stems are present, selection should achieve a target
density of 70 trees per acre (25 ft. square spacing), though most areas
of the stand are not currently stocked with enough potential crop trees
to achieve this target. Crop tree selection should emphasize on
retention of high-value stems of hard maple, yellow birch and black
cherry; secondarily, high-quality soft maple and paper birch may be
selected. Disease-free beech should also be retained (unless within
around 10-15 feet of a designated crop tree) and can be released or
partially released if competitors are of low- to average-quality.

\newpage



\end{document}
