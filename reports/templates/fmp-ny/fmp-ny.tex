\documentclass[]{tufte-handout}

% ams
\usepackage{amssymb,amsmath}

\usepackage{ifxetex,ifluatex}
\usepackage{fixltx2e} % provides \textsubscript
\ifnum 0\ifxetex 1\fi\ifluatex 1\fi=0 % if pdftex
  \usepackage[T1]{fontenc}
  \usepackage[utf8]{inputenc}
\else % if luatex or xelatex
  \makeatletter
  \@ifpackageloaded{fontspec}{}{\usepackage{fontspec}}
  \makeatother
  \defaultfontfeatures{Ligatures=TeX,Scale=MatchLowercase}
  \makeatletter
  \@ifpackageloaded{soul}{
     \renewcommand\allcapsspacing[1]{{\addfontfeature{LetterSpace=15}#1}}
     \renewcommand\smallcapsspacing[1]{{\addfontfeature{LetterSpace=10}#1}}
   }{}
  \makeatother

\fi

% graphix
\usepackage{graphicx}
\setkeys{Gin}{width=\linewidth,totalheight=\textheight,keepaspectratio}

% booktabs
\usepackage{booktabs}

% url
\usepackage{url}

% hyperref
\usepackage{hyperref}

% units.
\usepackage{units}


\setcounter{secnumdepth}{-1}

% citations

% pandoc syntax highlighting

% longtable
\usepackage{longtable,booktabs}

% multiplecol
\usepackage{multicol}

% strikeout
\usepackage[normalem]{ulem}

% morefloats
\usepackage{morefloats}


% tightlist macro required by pandoc >= 1.14
\providecommand{\tightlist}{%
  \setlength{\itemsep}{0pt}\setlength{\parskip}{0pt}}

% title / author / date
\date{}


\begin{document}





\pagenumbering{gobble} \vspace*{30pt}

\huge FOREST MANAGEMENT PLAN

\vspace{30pt}

\LARGE Hydefam Holdings Property

\vspace{22pt}

\normalsize Town of Harrietstown, Franklin County, New York

\vspace{70pt}

\small 72.4 acres

\vspace{18pt}

Parcel IDs:

\vspace{8pt}

457-4-16.00

467-1-1.00

467-1-25.00

467-1-26.000

467-1-28.000

467-1-29.00

\vspace{10pt}

Deed (Liber/Pages): 604/197, 604/199

\vspace{10pt}

480-a certification number: 16-033

\vspace{10pt}

Original certication date: 02/20/2004

\vspace{60pt}

Prepared by:

\includegraphics[width=0.5\linewidth,height=0.5\textheight]{C:/Users/Neal/projects/cruise/logo-name-small}

John D. Foppert \& Neal F. Maker\\
96 Durocher Road\\
Saranac, NY 12981

\vspace{30pt}

\large January 07, 2020

\pagebreak
\pagenumbering{arabic}

\section{Introduction}\label{introduction}

This plan covers the fifteen year period from 2019 to 2034. It lays out
the near- and medium-term actions that should guide the development of
the Hydefam Holdings forest. It also qualifies portions of the property
for continued enrollment in the 480-a Forest Tax Law program and
commensurate reduction in property taxes.\footnote{Further information
  about 480-a can be found at the New York Department of Conservation's
  website: \url{https://www.dec.ny.gov/lands/5236.html}. \vspace{20pt}}
Owners participating in the program are obliged to manage enrolled
portions of their property according to their approved forest management
plan and to make any reasonable investments for improvement that the
plan recommends. Its recommendations were developed in accordance with
the principles and practices of scientifically sound forestry, as
described in the relevant management guidelines, textbooks and academic
journals.

\section{Property Description}\label{property-description}

The Hydefam Holdings forest represents a well-managed resource
positioned to continue generating value on a sustainable basis. Eighty
eight percent of the 72.4 acre property is productive forestland to be
managed according to this plan. The property is located in the
Adirondack Park in the Town of Harrietstown, Franklin County, New York.
The property is accessed off of Kiwassa Lake Road and State Route 3. The
productive forestland is operationally accessible via a well developed
internal road network. Property-wide, elevations range from 1510 to 1580
feet above mean sea level. The property fronts Kiwassa Lake along
approximately 1/3 of a mile of shoreline or wetland complexes. There are
no mapped streams on the property, though several intermittent streams
follow established channels, each draining small areas of less than 100
acres. The property's boundaries are well established. A portion of the
property's boundary is defined by Kiwassa Lake, while most is defined by
an irregular boundary line which runs for over two miles. These lines
are generally well marked in the field (typically with tree blazes and
red boundary marking paint), but will require on-going maintenance.
Soils, forest health, and other pertinent topics are discussed in the
individual stand area descriptions that follow.

\section{Principles, Goals \& Strategies For Forest
Management}\label{principles-goals-strategies-for-forest-management}

\subsection{Conservation}\label{conservation}

The ecological functioning, productive capacity and biological diversity
of the forest resource should be maintained or improved over time so as
to provide opportunities for the current or future landowners to
continue to enjoy and use the property. A management strategy that is
sustainable in the long-term and viable in the short- and medium-terms
offers a strong measure of protection against future development or
conversion.

\subsection{Timber management}\label{timber-management}

Management should provide regular returns from timber harvesting.
Long-term value growth is provided by maintaining full site occupancy
with investment-grade stems: healthy trees capable of producing high
quality sawtimber or veneer and worth retaining in the stand until they
reach their full, site- and species-specific target diameters. Tree
species which yield sought-after, high-value wood should be promoted
within each stand or, when regenerating a new stand, attention should be
paid to providing the stand conditions which favor the establishment of
those species. At a property-wide scale, a variety of species should be
maintained, providing options for seizing future market opportunities
and a hedge against species-specific market depreciation. Among desired
species, additional preference should be given to individual trees of
sufficient vigor and grade-potential for strong future value growth.
Consideration of economic efficiency should inform the timing and
coordination of infrastructure investments and stand maintenance,
improvement and harvest operations.

\subsection{Amenity values}\label{amenity-values}

Conscientious management can create or maintain a landscape that is
attractive, accessible and conducive to reflection, exploration and
appreciation. Attractiveness can be managed for by fostering diversity
within the landscape: promoting the growth and development of the most
appealing individual trees in some places; maintaining the look, feel
and accompanying privacy provided by a dense forest in other places; and
elsewhere creating occasional vistas out from the forest and
improvements in depth of visual penetration within it. Carefully planned
and deliberately located infrastructure should facilitate the satisfying
use of the property, creating an appropriate balance between access and
connectedness, on the one hand, and places of refuge and sanctuary, on
the other. A system of roads and trails of various sizes, suited for
various purposes, and interconnected with a broader trail network,
provide for both enjoyable recreation and efficient operations.

\section{Stand Descriptions \& Management
Recommendations}\label{stand-descriptions-management-recommendations}

Presented below are detailed stand-by-stand descriptions of the forest,
the long-term structural, compositional and functional goals for each
stand, and the near-term silvicultural treatments or management
activities that have been prescribed to advance each stand toward those
goals. The data presented in the following pages was obtained from a
thorough inventory of the property in October of 2019. General
conditions were assessed qualitatively in conjunction with quantitative
sampling. Observational notes and sample summary statistics together
provide the basis for the stand descriptions and management
recommendations. All sampling was done using a systematic sample and
variable radius plots. In stands with uneven-aged structures, all trees
6" in diameter at breast height (dbh) and larger were measured in each
plot. In stands with even-aged structures, all main-canopy trees were
measured in each plot.

When contractors are used to implement silvicultural prescriptions, they
should be highly skilled, properly equipped, fully insured, and closely
supervised. A professional forester should prepare and administer
commercial treatments, and logging operations should be timed to
coincide with favorable weather conditions (working on wet soils only
when they are frozen, for instance) and favorable timber markets. The
dates assigned to timber harvests and other management activities
prescribed in this plan are intended to guide, rather than constrain,
forest management. To accommodate dynamic markets and variable weather,
scheduled timber harvests may be advanced or delayed by one year from
the date indentified in this plan; if operational or economic conditions
change substantially, the management schedule may be further revised by
an ammendment to this plan.

The property should be reassessed in 2024 and the findings brought to
bear on a reassessment of the goals and strategies proposed in this
plan, leading to a formal management plan update.

\newpage

\section{Management Schedule}\label{management-schedule}

\small 2020 - Management plan update; boundary line maintenance

\vspace{5pt}

\noindent 2021 - No scheduled activity

\vspace{5pt}

\noindent 2022 - No scheduled activity

\vspace{5pt}

\noindent 2023 - No scheduled activity

\vspace{5pt}

\noindent 2024 - No scheduled activity

\vspace{5pt}

\noindent 2025 - Management plan update; boundary line maintenance

\vspace{5pt}

\noindent 2026 - No scheduled activity

\vspace{5pt}

\noindent 2027 - No scheduled activity

\vspace{5pt}

\noindent 2028 - No scheduled activity

\vspace{5pt}

\noindent 2029 - No scheduled activity

\vspace{5pt}

\noindent 2030 - Management plan update; boundary line maintenance

\vspace{5pt}

\noindent 2031 - No scheduled activity

\vspace{5pt}

\noindent 2032 - No scheduled activity

\vspace{5pt}

\noindent 2033 - No scheduled activity

\vspace{5pt}

\noindent 2034 - Full management plan revision; boundary line
maintenance

\newpage

\section{Area 1}\label{area-1}

Mixedwood\\
\noindent 64.00 acres total\\
\noindent 6.40 acres ineligible wetlands\\
\noindent 2.00 acres ineligible roads and landings

\subsection{Site-specific information}\label{site-specific-information}

\begin{quote}
\begin{itemize}
\tightlist
\item
  \textbf{Soils:}\\
  \indent\indent Monadnock, Tunbridge, Adirondack
\end{itemize}
\end{quote}

\begin{quote}
\begin{itemize}
\tightlist
\item
  \textbf{Site Class:}\\
  \vspace{2pt} II (determined from soil mapping and field assessment)
\end{itemize}
\end{quote}

\begin{quote}
\begin{itemize}
\tightlist
\item
  \textbf{Access:}\\
  \vspace{2pt} Less than 1 mile
\end{itemize}
\end{quote}

\begin{quote}
\begin{itemize}
\tightlist
\item
  \textbf{Stand history:}\\
  \vspace{2pt} This stand has been continuously forested for at least
  the past 100 years. It appears to have been harvested on a recurring
  basis over most of that period, including a cut of mostly spruce in
  around the turn of the last century, maple and birch in the 1950's, a
  hemlock-focused harvest in the 1990's, and a silviculturlly prescribed
  harvest in 2018 and 2019. This recent harvest removed 32.175 MBF of
  hardwood logs, 21.57 MBF softwood logs, and 158 cords of firewood and
  pulp.
\end{itemize}
\end{quote}

\subsection{Current forest
information}\label{current-forest-information}

\begin{quote}
\begin{itemize}
\tightlist
\item
  \textbf{Age Class Structure:}\\
  \vspace{2pt} Uneven-aged\\

  \begin{marginfigure}
  \includegraphics{fmp-ny_files/figure-latex/unnamed-chunk-2-1} \caption[Distributions are approximated with kernel density estimation]{Distributions are approximated with kernel density estimation. Common species are those that account for at least 8 percent of the total stocking and areas under each curve represent species basal areas.}\label{fig:unnamed-chunk-2}
  \end{marginfigure}
\end{itemize}
\end{quote}

\begin{quote}
\begin{itemize}
\tightlist
\item
  \textbf{Species (\% stocking):}\\
  \vspace{2pt} hemlock (25\%), hard maple (19\%), yellow birch (17\%),
  soft maple (12\%), cedar (10\%), beech (5\%), ash (4\%), black cherry
  (4\%), spruce (2\%), aspen (1\%), fir (1\%)
\end{itemize}
\end{quote}

\begin{quote}
\begin{itemize}
\tightlist
\item
  \textbf{Regeneration:}\\
  \vspace{2pt} Regeneration is present throughout the stand. It
  generally reflects the composition of the overstory cohort, including
  balsam fir, beech, sugar maple, red spruce, and other species , though
  beech is somewhat overrepresented. Around 40\% of plots sampled
  contained at least some sugar maple saplings overtopping neighboring
  competiors.
\end{itemize}
\end{quote}

\begin{quote}
\begin{itemize}
\tightlist
\item
  \textbf{Forest health:}\\
  \vspace{2pt} Few forest health issues were observed on this portion of
  the property. Beech bark disease is present and affects older beech
  trees. Windthrow will be a persistent concern in portions of the stand
  where soils are wet or shallow-to-bedrock.
\end{itemize}
\end{quote}

\begin{quote}
\begin{itemize}
\tightlist
\item
  \textbf{Size class structure (\%BA):}\\
  \vspace{2pt} \indent 6-10'': 17\% \textbar{} 11-16'': 32\% \textbar{}
  17-22'': 38\% \textbar{} 23"+: 12\%
\end{itemize}
\end{quote}

\subsection{Inventory information}\label{inventory-information}

\begin{quote}
\begin{itemize}
\tightlist
\item
  8 points, 10 BAF, October, 2019
\end{itemize}
\end{quote}

\begin{figure}
\includegraphics{fmp-ny_files/figure-latex/unnamed-chunk-5-1} \caption[Points represent individual plots]{Points represent individual plots. Asterisk represnts stand average. Radial lines are quadratic stand diameters.}\label{fig:unnamed-chunk-5}
\end{figure}

\begin{longtable}[]{@{}lrr@{}}
\caption{Measures of stocking for all live trees (total) and acceptable
growing stock.}\tabularnewline
\toprule
& Total & Acceptable\tabularnewline
\midrule
\endfirsthead
\toprule
& Total & Acceptable\tabularnewline
\midrule
\endhead
Basal area (sqft/ac) & 101 & 86\tabularnewline
QSD (in) & 13 & 13\tabularnewline
Stems/ac & 109 & 92\tabularnewline
\bottomrule
\end{longtable}

\subsection{Silvicultural
prescription}\label{silvicultural-prescription}

No silvicultural treatment is called for at this time.

\newpage



\end{document}
