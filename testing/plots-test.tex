\documentclass[]{article}
\usepackage{lmodern}
\usepackage{amssymb,amsmath}
\usepackage{ifxetex,ifluatex}
\usepackage{fixltx2e} % provides \textsubscript
\ifnum 0\ifxetex 1\fi\ifluatex 1\fi=0 % if pdftex
  \usepackage[T1]{fontenc}
  \usepackage[utf8]{inputenc}
\else % if luatex or xelatex
  \ifxetex
    \usepackage{mathspec}
  \else
    \usepackage{fontspec}
  \fi
  \defaultfontfeatures{Ligatures=TeX,Scale=MatchLowercase}
\fi
% use upquote if available, for straight quotes in verbatim environments
\IfFileExists{upquote.sty}{\usepackage{upquote}}{}
% use microtype if available
\IfFileExists{microtype.sty}{%
\usepackage{microtype}
\UseMicrotypeSet[protrusion]{basicmath} % disable protrusion for tt fonts
}{}
\usepackage[margin=1in]{geometry}
\usepackage{hyperref}
\hypersetup{unicode=true,
            pdftitle={Plot Conditions},
            pdfauthor={Neal Maker},
            pdfborder={0 0 0},
            breaklinks=true}
\urlstyle{same}  % don't use monospace font for urls
\usepackage{graphicx,grffile}
\makeatletter
\def\maxwidth{\ifdim\Gin@nat@width>\linewidth\linewidth\else\Gin@nat@width\fi}
\def\maxheight{\ifdim\Gin@nat@height>\textheight\textheight\else\Gin@nat@height\fi}
\makeatother
% Scale images if necessary, so that they will not overflow the page
% margins by default, and it is still possible to overwrite the defaults
% using explicit options in \includegraphics[width, height, ...]{}
\setkeys{Gin}{width=\maxwidth,height=\maxheight,keepaspectratio}
\IfFileExists{parskip.sty}{%
\usepackage{parskip}
}{% else
\setlength{\parindent}{0pt}
\setlength{\parskip}{6pt plus 2pt minus 1pt}
}
\setlength{\emergencystretch}{3em}  % prevent overfull lines
\providecommand{\tightlist}{%
  \setlength{\itemsep}{0pt}\setlength{\parskip}{0pt}}
\setcounter{secnumdepth}{0}
% Redefines (sub)paragraphs to behave more like sections
\ifx\paragraph\undefined\else
\let\oldparagraph\paragraph
\renewcommand{\paragraph}[1]{\oldparagraph{#1}\mbox{}}
\fi
\ifx\subparagraph\undefined\else
\let\oldsubparagraph\subparagraph
\renewcommand{\subparagraph}[1]{\oldsubparagraph{#1}\mbox{}}
\fi

%%% Use protect on footnotes to avoid problems with footnotes in titles
\let\rmarkdownfootnote\footnote%
\def\footnote{\protect\rmarkdownfootnote}

%%% Change title format to be more compact
\usepackage{titling}

% Create subtitle command for use in maketitle
\newcommand{\subtitle}[1]{
  \posttitle{
    \begin{center}\large#1\end{center}
    }
}

\setlength{\droptitle}{-2em}

  \title{Plot Conditions}
    \pretitle{\vspace{\droptitle}\centering\huge}
  \posttitle{\par}
    \author{Neal Maker}
    \preauthor{\centering\large\emph}
  \postauthor{\par}
      \predate{\centering\large\emph}
  \postdate{\par}
    \date{September 22, 2018}


\begin{document}
\maketitle

\begin{verbatim}
## Warning: package 'tidyverse' was built under R version 3.5.1
\end{verbatim}

\begin{verbatim}
## -- Attaching packages -------------------------------------------- tidyverse 1.2.1 --
\end{verbatim}

\begin{verbatim}
## v ggplot2 2.2.1     v purrr   0.2.5
## v tibble  1.4.2     v dplyr   0.7.5
## v tidyr   0.8.1     v stringr 1.3.1
## v readr   1.1.1     v forcats 0.3.0
\end{verbatim}

\begin{verbatim}
## Warning: package 'tidyr' was built under R version 3.5.1
\end{verbatim}

\begin{verbatim}
## Warning: package 'readr' was built under R version 3.5.1
\end{verbatim}

\begin{verbatim}
## Warning: package 'forcats' was built under R version 3.5.1
\end{verbatim}

\begin{verbatim}
## -- Conflicts ----------------------------------------------- tidyverse_conflicts() --
## x dplyr::filter() masks stats::filter()
## x dplyr::lag()    masks stats::lag()
\end{verbatim}

\begin{verbatim}
## Warning: package 'ggthemes' was built under R version 3.5.1
\end{verbatim}

\begin{verbatim}
## Parsed with column specification:
## cols(
##   spp = col_character(),
##   stand = col_integer(),
##   plot = col_integer(),
##   code = col_integer(),
##   dbh = col_double(),
##   logs = col_integer(),
##   cond = col_integer(),
##   cut = col_integer()
## )
\end{verbatim}

\begin{verbatim}
## # A tibble: 13 x 17
##     plot stand ba_live ba_ags ba_inv ba_crop ba_snags tpa_live tpa_ags
##    <int> <int>   <dbl>  <dbl>  <dbl>   <dbl>    <dbl>    <dbl>   <dbl>
##  1   305     2     150    100     20       0        0      197      90
##  2   306     2     140    120     50      20        0      148     113
##  3   307     2      90     90     50       0        0      297     297
##  4   308     2     100     70     40       0        0      246     205
##  5   309     2     190    130     40      20        0      213      94
##  6   312     2     120    120     70      30        0      176     176
##  7   403     2      70     60     50       0        0      111      93
##  8   404     2      60     60     20      10        0       87      87
##  9   405     2      90     90     30       0        0      133     133
## 10   409     2     180    130     20       0        0      141      95
## 11   410     2     270    190     70      30        0      204     109
## 12   411     2     120     50     50       0        0      232      78
## 13   412     2     110     60     10       0        0      118      60
## # ... with 8 more variables: tpa_inv <dbl>, tpa_crop <dbl>,
## #   tpa_snags <dbl>, qsd_live <dbl>, qsd_ags <dbl>, qsd_inv <dbl>,
## #   qsd_crop <dbl>, qsd_snags <dbl>
\end{verbatim}

\subsection{R Markdown}\label{r-markdown}

This is an R Markdown document. Markdown is a simple formatting syntax
for authoring HTML, PDF, and MS Word documents. For more details on
using R Markdown see \url{http://rmarkdown.rstudio.com}.

When you click the \textbf{Knit} button a document will be generated
that includes both content as well as the output of any embedded R code
chunks within the document. You can embed an R code chunk like this:

\begin{verbatim}
##      speed           dist       
##  Min.   : 4.0   Min.   :  2.00  
##  1st Qu.:12.0   1st Qu.: 26.00  
##  Median :15.0   Median : 36.00  
##  Mean   :15.4   Mean   : 42.98  
##  3rd Qu.:19.0   3rd Qu.: 56.00  
##  Max.   :25.0   Max.   :120.00
\end{verbatim}

\subsection{Including Plots}\label{including-plots}

You can also embed plots, for example:

\includegraphics{plots-test_files/figure-latex/pressure-1.pdf}

Note that the \texttt{echo\ =\ FALSE} parameter was added to the code
chunk to prevent printing of the R code that generated the plot.


\end{document}
