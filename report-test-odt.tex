\documentclass[]{tufte-handout}

% ams
\usepackage{amssymb,amsmath}

\usepackage{ifxetex,ifluatex}
\usepackage{fixltx2e} % provides \textsubscript
\ifnum 0\ifxetex 1\fi\ifluatex 1\fi=0 % if pdftex
  \usepackage[T1]{fontenc}
  \usepackage[utf8]{inputenc}
\else % if luatex or xelatex
  \makeatletter
  \@ifpackageloaded{fontspec}{}{\usepackage{fontspec}}
  \makeatother
  \defaultfontfeatures{Ligatures=TeX,Scale=MatchLowercase}
  \makeatletter
  \@ifpackageloaded{soul}{
     \renewcommand\allcapsspacing[1]{{\addfontfeature{LetterSpace=15}#1}}
     \renewcommand\smallcapsspacing[1]{{\addfontfeature{LetterSpace=10}#1}}
   }{}
  \makeatother

\fi

% graphix
\usepackage{graphicx}
\setkeys{Gin}{width=\linewidth,totalheight=\textheight,keepaspectratio}

% booktabs
\usepackage{booktabs}

% url
\usepackage{url}

% hyperref
\usepackage{hyperref}

% units.
\usepackage{units}


\setcounter{secnumdepth}{-1}

% citations

% pandoc syntax highlighting

% longtable
\usepackage{longtable,booktabs}

% multiplecol
\usepackage{multicol}

% strikeout
\usepackage[normalem]{ulem}

% morefloats
\usepackage{morefloats}


% tightlist macro required by pandoc >= 1.14
\providecommand{\tightlist}{%
  \setlength{\itemsep}{0pt}\setlength{\parskip}{0pt}}

% title / author / date
\title{Nalley Property Forest Management Plan}
\author{Pekin Branch Forestry}
\date{October 21, 2018}


\begin{document}

\maketitle




\section{Signature Page}\label{signature-page}

\textbf{Owner:}\\
Mark \& Carla Nalley\\
111 Baytree Court\\
Winter Springs, FL. 32708

\textbf{Property:}\\
194 acres, dwelling, studio \& barn\\
Parcel ID \# 03-47\\
SPAN \# 780-248-10145\\
Map delineation based on VMP Photo \#160212

\textbf{Plan preparer:}\\
Neal F. Maker \& John D. Foppert\\
Pekin Branch Forestry\\
1324 West County Road\\
Calais, VT 05648\\
(802) 229-9757

\textbf{Effective date of plan:} April 1, 2019

line

line

I certify that the herein described forest land, exclusive of any house
site or other developed portion, is at least 25 acres in size and is
under active long-term forest management for the purpose of growing and
harvesting repeated forest crops in accordance with minimum acceptable
standards for forest management. These management standards include
following the practices outlined in the booklet ``Acceptable Management
Practices for Maintaining Water Quality on Logging Jobs in Vermont'' in
order to control stream siltation and soil erosion.

By signing below, I agree to manage my forestland according to the
current, approved plan.

line

\_\_\_\_\_\_\_\_\_\_\_\_\_\_\_\_\_\_\_\_\_\_\_\_\_\_\_\_\_\_\_\_ Date:
\_\_\_\_\_\_\_\_\_\_\_\_\_\_\\
Mark L. Nalley

line

\_\_\_\_\_\_\_\_\_\_\_\_\_\_\_\_\_\_\_\_\_\_\_\_\_\_\_\_\_\_\_\_ Date:
\_\_\_\_\_\_\_\_\_\_\_\_\_\_\\
Carla M. Nalley

line

line

This ten year forest management plan for the 194 acre property owned by
Mark and Carla Nalley and located in Woodbury, Vermont, meets the
standards promulgated by the Vermont Department of Forests, Parks and
Recreation as required for eligibility in the Use Value Appraisal
Program.

line

\_\_\_\_\_\_\_\_\_\_\_\_\_\_\_\_\_\_\_\_\_\_\_\_\_\_\_\_\_\_\_\_ Date:
\_\_\_\_\_\_\_\_\_\_\_\_\_\_\\
Washington County Forester

\subparagraph{pagebreak}\label{pagebreak}

\section{Introduction}\label{introduction}

A forest management plan is a blueprint for responsible land
stewardship. It is the result of a planning process that incorporates an
assessment of the history and current conditions on the property,
consideration of the various courses of future development that the
forest could follow, and discernment with the landowners as to which
outcomes best suit their particular objectives. This management plan is
organized to reflect that process. It begins with a description of the
property and its setting, focusing on attributes that should be
considered in the course of planning. It then outlines the landowners'
specific objectives for the property before recommending management
actions that will help achieve those objectives.

Covering the ten year period from 2019-2028, this management plan is a
blueprint for the near- and medium-term actions that should guide the
development of the Nalley forest. This management plan also qualifies
the property for Use Value Appraisal and commensurate reduction in
property taxes. Owners participating in the Use Value Appraisal program
are obliged to manage their property according to the plan and to make
any reasonable investments for improvement that the plan recommends. Its
recommendations were developed in accordance with the principles and
practices of scientifically sound forestry, as described in the relevant
management guidelines, textbooks and academic journals.

\section{Location \& Description}\label{location-description}

The 194 acre Nalley property is 92 percent wooded with the remainder
used as a field, pond, and house site. It is located in the eastern part
of the town of Woodbury, bordered by the Cabot Road on the north.
Elevations range from 1360 feet along the bed of Jug Brook in the east
to 1500 feet in the southwest, and slopes are mostly gentle, facing
northeast to southeast. The property's productive timberland can
naturally be divided into four distinct stands, defined by tree species
composition, structure and land-use history.

The land was first enrolled in the state's Use Value Appraisal program
in 1999 and has been managed with a plan since then. Old wire fences,
stone walls and blazes mark many of the property lines, though the
boundaries are unclear in some places.

\section{Physical Geography \&
Environment}\label{physical-geography-environment}

\subsection{Bedrock geology}\label{bedrock-geology}

The underlying geology of Vermont is naturally organized into several
north-northeast trending bedrock belts. Each belt is made up of rocks
generally of the same age and formed by similar processes. The Nalley
property lies within the Waits River formation, which runs through the
broad Connecticut Valley Trough, east of the Green Mountains. Rocks of
the Waits River formation were originally formed in the bed of the warm,
shallow Iapetus Ocean, which was teeming with life. The shells of marine
animals were mixed with sediments on the ocean floor and compressed to
form calcium rich limestone, which was later metamorphosed when the
microcontinent Avalonia collided with the Laurasian supercontinent
during the Acadian Orogeny, some 325-375 million years ago. Intermixed
mudstones were also metamorphosed and upended at that time, becoming
flakey phyllite bands amid the crumbly limestones. As these rocks have
since weathered and been broken up, their mineralogy has greatly
influenced the region's soils and waters, and has largely defined the
natural and human communities that we now see.

\subsection{Glaciation}\label{glaciation}

For hundreds of millions of years following the Avalonian collision,
not-yet-Vermont's rugged terrain was washed with rain and wind and worn
down to gentile, rolling hills and mountains. The weathered rock and
organic deposits collected under the force of water, wind and gravity to
form deep, well developed soils like those still common in the
Southeastern US. But some two million years ago a long episode of frigid
temperatures began, which led to massive accumulations of snow in the
north. The snow's increasing weight compressed lower (older) layers into
crystalline ice and, eventually, to a deformed, plasma-like ice
material, which grew to over a mile thick and spread under its own
weight like pancake batter. Four of these glaciers developed in
succession, each spreading south, then melting down when the weather
warmed, only to build up again as it cooled. The glaciers slid over,
scraped across, or ground down the bedrock and soil in front of and
beneath them, thoroughly removing and mixing the existing soils.

As the most recent glacier (the Wisconsin) melted and retreated north,
the unsorted silts, sands, gravel, and boulders that it had picked up
and mixed together were dumped in place, blanketing the land in glacial
till. These are the soils that remain on the Nalley property, and across
much of Woodbury. However, the huge volumes of water produced by the
melting glacier formed great rivers, which washed through the till,
sorting the finer particles out from coarser ones, depositing them in
bands along their margins, and washing much of the material out to sea.
In places, as the glacier receded from river and tributary valleys,
impounded meltwater formed deep, pro-glacial lakes, filling the basins
they formed in until they spilled over an outlet at the low point on the
watershed divide. As successively lower outlets were exposed by the
retreating ice sheet, lake levels would drop, streams and rivers would
cut through older landforms, and new lacustrine features would be built.
Many of these can be seen to the east of the property, along the
Kingsbury Branch and along the main Winooski River, which were at one
time inundated by glacial Lake Winooski.

\subsection{Soils}\label{soils}

Today's soils owe much of their chemical makeup to the underlying
bedrock; and much of their structure, depth, and texture to the glacial
processes that laid them down. The calcium rich mineral content of the
region's limestones buffers acidity well, and is largely responsible for
the rich northern hardwood forests and fens for which the Vermont
Piedmont region is famous.

Soils on the property are almost all dense tills, whose lower strata
were compressed by the weight of the glaciers. They contain a mix of
sand, silt, clay and stones. In lower drainageways and concavities on
the property, the dense lower soil layer traps water and more poorly
drained Cabot silt loams are found. Elsewhere, groundwater is able to
flow down and sideways on top of the dense soil layer and we see
somewhat better drained Buckland silt loams.

Tree growth depends on many factors, including micro-site conditions,
nearby vegetation, and light availability; but soil classification is
still one of the best ways to broadly predict growth, productivity, and
suitability of a site for certain species. On the Nalley property, the
Buckland soils are generally the most productive, and support our
`pickier' trees like sugar maple and white ash. Their looser upper
strata are moderatly deep, allowing for fairly good rooting, and they
provide plenty of water without really depriving trees of soil oxygen.
Cabot soils are quite similar compositionally, but their upper strata
are not as deep and their poor drainage restricts trees of oxygen,
slowing growth.

\subsection{Hydrology}\label{hydrology}

Jug Brook bisects the property, flowing west to east, and several of its
tributaries cross the property as well. Two of them cross the southern
section of the property and join in Area 3, while a third originates in
a pond just behind the house and flows south through Area 2. The
groundwater is close to the surface in many areas of the property as
well, making for soft soils.

\section{Ecologic \& Anthropogenic
History}\label{ecologic-anthropogenic-history}

\subsection{Historic ecological
context}\label{historic-ecological-context}

Immediately after deglaciation, the bare landscape that was to become
Vermont was colonized, if only sparsely, by tundra lichens, grasses,
sedges, and forbs. Within a few hundred years, this vegetative ground
cover was probably close to continuous, with some shrub willows and
dwarf birches scattered around. Over the next few hundred years the
shrub component grew denser, with alder migrating onto the landscape and
patches of white and black spruce and tree-sized birch becoming
established. This tundra-forest transition zone provided habitat
for---and was probably partly maintained by---muskoxen, mastadons,
caribou, giant beavers, and other megafauna.

Over the course of centuries, the transition from open tundra to closed
boreal forest continued. For the millennium or two that boreal forests
dominated the landscape, species composition was continually shifting as
the climate warmed and then cooled and as patterns of rainfall
fluctuated, with competitive dynamics and disturbance regimes changing
in response. Fire-maintained jackpine forests expanded and then declined
relative to spruce; tammarack and, later, balsam fir became increasingly
important; and, by around 10,000 years ago, white pine, hemlock, and
various hardwoods were becoming more common, marking the start of a
transition to a temperate mixedwoods forest. By this time, many of the
easily quarried megafauna had already been hunted to extinction.

By 9,000 years ago elm, ash, birch, beech, maple, and apparently oak
were well established; with white pine and hemlock, these hardwoods
marked the full transition from boreal to mixedwoods. Spruce (mostly red
spruce) and balsam fir were still present on the landscape, but were
restricted to colder or wetter sites where other species were
uncompetitive. The fauna then were broadly similar to those we are
familiar with today: moose, deer, black bears, beavers, wolves, mountain
lions, and most of the birds, amphibians, and small mammals now
associated with northern mixed forests.

By around 7,000 years ago, biomes---the most general level of ecological
classification (e.g.~tundra, boreal forest, temperate forest)---had
arrived at or very near their modern distributions. Nevertheless,
species composition and arrangement at finer spatial scales remained
dynamic. The following few thousand years continued to see change in the
forest: white pine, which had once been the most widespread species,
became increasingly less prevalent as maple, yellow birch, and
especially hemlock occupied more and more of the landscape. By around
5,000 years ago, hemlock was the singularly dominant tree species across
nearly all of New England and the northeast. Quite suddenly, however, it
was nearly eliminated from the landscape. A disease or insect (probably
the hemlock looper) was almost certainly the proximate cause of the
decline, but it may have functioned in conjunction with a centuries-long
drying trend and discrete periods of severe drought. Regardless of the
details of its extirpation, hemlock did not return to the landscape in
any meaningful way for 2,000-3,000 years, and while it did later become
a very important component of the northern forest, it never returned to
as extensive and dominant a position as it once occupied.

Even in the centuries preceding European settlement, region-wide changes
were observed in the forest. Within the boreal-deciduous continuum on
which the temperate mixedwoods lie, species composition began to skew
more towards the boreal. Spruce and fir dominated stands became more
prevalent on the landscape, and spruce became more abundant as an
associate within stands dominated by other species.

In the millennia since deglaciation, biome-scale landscape changes have
closely tracked global- or continental-scale change in the climate,
though generally following a lag of a few to several hundred years.
Within-biome changes have been similarly responsive to climactic
drivers, but these have been compounded by complex, finer-scale
landscape and stand dynamics involving differential growth and
reproduction strategies, interspecific competition, catastrophic weather
events, insect outbreaks, wildlife interactions, and human land use.

\subsection{Human land-use history}\label{human-land-use-history}

The earliest human settlers entered the landscape fairly soon after
deglaciation, following the grazing herds of game through the uplands
and establishing seasonal campsites on the shores of the Champlain Sea,
which then occupied the Champlain and St.~Lawrence valleys. These
paleoindian groups became increasingly effective at hunting the
megafunana, such as mastadons, stag elk-moose, giant ground sloths,
giant beavers, woodland muskox, and others, which were poorly adapted to
the predatory pressure humans applied. By around 11,000 years ago most
of these species had been hunted to extinction, and, as caribou herds
moved north with the expanding tundra and encroaching temperate
woodlands, paleoindian hunting tools were adapted to the smaller and
often fleeter game that was then available.

The cultural practices and social structure of the paleoindian groups
are not well understood, but it is clear that from the time the region
was first settled humans hunted, fished, and collected edible and
medicinal plants in the valleys and uplands of northern Vermont. Gradual
technological development and social ordering led to the emergence of a
discernibly different culture around 8,000 years ago. The people of this
``woodland period'' moved between seasonal hunting and fishing camps,
but began to stay within a more circumscribed area than their
semi-nomadic predecessors. Established sites in lake and river valleys
were surrounded by upland hunting grounds and connected by networks of
waterways and portages. Mobility remained a more important survival
strategy than modifying the environment and the relatively low-density
population of this period did not significantly influence
landscape-scale forest dynamics, especially in upland areas, where
hunters or travelers only occasionally visited.

Eventually, inhabitants (who were by that time known as the Wabenaki)
began to more actively manage and manipulate the natural environment by
tending wild plants for food and medicine, gathering firewood, and
setting forest fires to control species compositions and encourage game
animals. Those effects were focused on the areas where their camps and
villages were concentrated, however, and had little effect on upland
areas like Woodbury.

When European-Americans began settling the region in the mid to late
1700s, however, they quickly started to take advantage of the
calcium-rich soils that are naturally present. Following a period of
subsistence farming, the Vermont Piedmont became known first for wheat
production, then for sheep farming---providing wool to textile factories
to the south---and later for dairy farming. Vermont was close enough to
Boston and other large coastal populations to ship milk to them without
it spoiling and the climate and soils made for good pasture. By the late
1800s, most of the property was probably cleared and used as pasture or
hay field. More accessible, level areas around the house and old
homestead may have been used to grow crops at that time. Scattered very
large, branchy sugar maples in Area 2 could indicate an old sugarbush
there, or they may have grown in mostly open pasture.

With the advent of refrigeration and the development of better
transportation networks in the second half of the 19th century, much of
the country's agricultural production moved to the fertile Midwest and a
period of agricultural decline began throughout Vermont. Many pastures
and hay fields were abandoned, including those on the Nalley property,
and as a testament to the land's resilience, quickly regrew trees.
Forest has since become the region's dominant landcover, but the land
still shows signs of its agricultural past. Soil compaction from plowing
facilitated the growth of white pine on more intensively used sites in
the Piedmont (like Areas 1 and 4 on the Nalley property) and decreased
the abundance of other species like beech and hemlock. Forested pastures
and other less intensively used sites often regrew beech and hemlock,
but still saw a shift toward less shade-tolerant species like ash and
birch. And many old sugarbushes now host new hardwood forests with
scattered, gnarly ``legacy'' maples remaining.

In the decades following agricultural abandonment, old fields hosted
young trees that were largely ignored. As the trees reached commercial
sizes, however, landowners began cutting them; often opportunistically
picking out the highest value stems. This seems to have been the pattern
on the Nalley property, where high quality sawtimber is now quite
sparse. The last logging operation took place just after the property
was enrolled in the state's Current Use program, and was probably done
in a more forward-looking manner. One of its results was to establish an
abundance of young trees (especially in Area 2) which will form the
basis for a more valuable, carefully tended forest in the future.

\section{Current Conditions}\label{current-conditions}

\subsection{Landscape context}\label{landscape-context}

In addition to driving forest development patterns, its agricultural
history left a fairly widespread network of roads across the Northern
Vermont Piedmont and large blocks of unbroken forest are now uncommon. A
dense network of streams, ponds, and wetlands is also present, though,
and together with the diverse forests and remaining fields, it supports
many different animal species.

The Nalley property is located in a fairly large forest block that
extends between Cabot Road, Route 14, and East Hill Road, and is home to
many different species of wildlife, including at least some of the
far-ranging, forest interior species like black bears and great horned
owls. The block has fewer ponds than other forest blocks nearby, and the
water features that do exist are probably especially important for
wildlife that live here. Jug Brook likely serves as a major corridor for
birds and mammals of all sizes, and small ponds like the one behind the
Nalleys' house provide them with drinking water, fishing grounds, and
nesting spots.

\subsection{Forest health}\label{forest-health}

A number of minor tree diseases and pests were observed on the property,
which affect individual trees but are not a major concern overall. These
include eutapella canker, maple borer, and white pine blister rust. Pine
weevil killed the leaders on many of the property's white pines when
they were young, which caused them to grow multiple leaders, diminishing
their timber value and putting them at higher risk of breakage.

Invasive bush honeysuckle was found growing throughout the property as
well. When present in high numbers, it can interfere with the
establishment of desirable tree regeneration. The honeysuckle is more
common closer to the road and pond, and will be described in individual
stand descriptions later in this plan.

\subsection{Cultural resources}\label{cultural-resources}

Two foundations and a spring in the southern part of the property are
all that is left of an old homestead there. The structures were probably
abandoned around the close of the 19th century. A number of beautiful
fieldstone walls around the Nalleys' house on the Cabot Road probably
date to the 19th century too, as does the house itself, which has
remained continually occupied.

\subsection{Access \& Operability}\label{access-operability}

A good landing area is available west of the house on the Cabot Road,
and a network of trails runs from there through the front half of the
property, providing good recreational access and relatively good
operational access. Care will need to be taken during logging operations
to maintain the trails in good condition for recreation. There is not
currently access to the southern half of the property, behind Jug Brook.
At one time, McCarty Road in the southeast extended into that area, but
it was abandoned years ago and its old course passes through private
properties. Access will probably be easier to develop by installing a
stream crossing over Jug Brook.

\section{Principles, Goals \& Strategies For Forest
Management}\label{principles-goals-strategies-for-forest-management}

\subsection{Conservation}\label{conservation}

The ecological functioning, productive capacity and biological diversity
of the forest resource shall be maintained or improved over time so as
to provide opportunities for the current or future landowners to
continue to enjoy and use the property. A management strategy that is
sustainable in the long-term and viable in the short- and medium-terms
offers a strong measure of protection against future development or
conversion.

\subsection{Ecological integrity, wildlife habitat, \&
biodiversity}\label{ecological-integrity-wildlife-habitat-biodiversity}

Management should prioritize the protection of critical ecological
functions, water resources, and threatened or rare plant and wildlife
communities. Relatively continuous forest cover should be maintained for
the soil protection, nutrient cycling, and hydrological regulation it
provides. Seeps, springs, and stream-side riparian zones should be
carefully delineated and protected. Management should also give
consideration to the habitat needs of native wildlife populations and to
relationship between the property, its neighbors and the larger
landscape they are nested within. Management should be informed by and
aim to improve landscape diversity, travel corridors, and habitat
connectivity. Locally under-represented habitat types should be
identified and promoted. Stand scale and sub-stand scale management
should focus on developing or maintaining species-specific habitat
needs, such as nesting sites, cover, mast production, preferred browse
or other unique structural and compositional requirements.

\subsection{Timber management}\label{timber-management}

Management should provide regular returns from timber harvesting.
Long-term value growth is provided by maintaining full site occupancy
with healthy trees capable of producing high quality sawtimber and
veneer. Tree species which yield sought-after, high-value wood shall be
promoted within each stand or, when regenerating a new stand, attention
shall be paid to providing the stand conditions which favor the
establishment of those species. At a property-wide scale, a variety of
species shall be maintained to provide opportunities to exploit future
market opportunities and as a hedge against species-specific market
depreciation. Among desired species, additional preference shall be
given to individual trees of sufficient vigor and grade-potential for
strong future value growth. Consideration of economic efficiency should
inform the timing and coordination of infrastructure investments and
stand maintenance, improvement and harvest operations.

\section{Stand Descriptions \& Management
Recommendations}\label{stand-descriptions-management-recommendations}

Presented below are detailed stand-by-stand descriptions of the forest,
the long-term structural, compositional and functional goals for each
stand, and the near-term silvicultural treatments or management
activities that have been prescribed to advance each stand toward those
goals. The data presented in the following pages was obtained from a
field examination of the property in September of 2018. General
conditions were assessed qualitatively in conjunction with quantitative
sampling of the overstory strata in each productive stand (Areas 1-4).
Observational notes and sample summary statistics together provide the
basis for the area descriptions and management recommendations.

Over the course of the management period for which this plan has been
written, work has been prescribed for a number of areas throughout the
forest. Much of the work is commercial-scale and it is expected that a
highly skilled, properly equipped, fully insured, closely supervised
contractor will carry out that work. A professional forester should
prepare and administer commercial treatments, and logging operations
should be timed to coincide with favorable weather conditions (working
on wet soils only when they are frozen, for instance) and favorable
timber markets. Use Value Appraisal program guidelines allow any
management activities prescribed in this plan to be carried out up to
three years before or after the date indicated. Landowners in the Use
Value Appraisal program must file a Forest Management Activity Report
with the County Forester by February 1st if any commercial logging
occurred in the previous year.

2028 will mark ten years since the inventory data on which this plan has
been based was collected; the property should be reinventoried then and
the findings brought to bear on a reassessment of the goals and
strategies proposed in this plan, leading to a formal management plan
update. At any point over the course of this management period, this
plan may be updated to incorporate that new data and reflect any new
thoughts, concerns or considerations on the part of the family or the
foresters helping to manage their land.

\subsection{Management schedule}\label{management-schedule}

\textbf{2021:}

Area 1: Shelterwood establishment cut\\
Area 2: Group selection conversion\\
Area 3: Group selection conversion\\
Area 4: Crown thinning

\textbf{2028:}

Reinventory property for 2029 plan update

\subparagraph{pagebreak}\label{pagebreak-1}

\section{Stand 1}\label{stand-1}

\textbf{White Pine}\\
\textbf{16.25 legal acres -- 18.14 measured acres}

\subsection{Stand conditions}\label{stand-conditions}

Stand 1 occupies an abandoned pasture along the Cabot Road west of the
house and is dominated by white pine. The site has generally gentile
terrain, which slopes toward the pond in the east; and the soils are a
mix of Cabot (in the wetter east) and Buckland (in the dryer west).

Many of the pines were subject to weevil damage early in life and have
multiple leaders, diminishing their timber volumes and quality and
putting them at higher risk of breakage. This accounts for the dearth of
investment-grade growing stock in the stand. Most of the weeviled trees
do have some sawtimber, however, and are considered acceptable growing
stock. According to the previous management plan, a thinning in 2000
removed much of the unacceptable growing stock that was present at that
time, which wouldn't have had any sawtimber at all.

There is little advance regeneration in the area, though some sugar
maple, balsam fir, white ash, and red maple stems are present. Deer are
browsing the more palatable sugar maple seedlings. Invasive honeysuckle
plants are also scattered throughout the stand, and should be taken into
consideration in any regeneration harvest.

\subsection{Quantitative stand data}\label{quantitative-stand-data}

\textbf{Site Class: } II (soils mapping \& field assessment)\\
\textbf{Access Distance: } Less than one mile\\
\textbf{Cruise Data: } 3 points, 10 BAF, Sep. 2018\\
\textbf{Age Class Structure: } Even-aged (\textasciitilde{}90 yrs old)\\
\textbf{Volume/ac: } 0 MBF veneer, 11.1 MBF sawtimber, 4.6 MBF tie logs,
\& 9 cds pulp

\textbf{Size Class Structure (\%BA): } 6-10'': 8\% --- 10-16'': 13\% ---
16-22'': 55\% --- 22+'': 24\%

line

\begin{longtable}[]{@{}lrrr@{}}
\caption{Measures of stocking for all live trees in the main canopy
(total), acceptable growing stock, and inventory-grade growing stock
(which is a subset of acceptable growing stock).}\tabularnewline
\toprule
& Total & Acceptable growing stock & Inventory-grade\tabularnewline
\midrule
\endfirsthead
\toprule
& Total & Acceptable growing stock & Inventory-grade\tabularnewline
\midrule
\endhead
Basal area (sqft/ac) & 127 & 100 & 23\tabularnewline
QSD (in) & 17 & 16 & 16\tabularnewline
Stems/ac & 84 & 69 & 17\tabularnewline
\bottomrule
\end{longtable}

\begin{longtable}[]{@{}lr@{}}
\caption{Species present and their percentage of total stand basal
area.}\tabularnewline
\toprule
Species & \% BA\tabularnewline
\midrule
\endfirsthead
\toprule
Species & \% BA\tabularnewline
\midrule
\endhead
white pine & 92\tabularnewline
soft maple & 8\tabularnewline
\bottomrule
\end{longtable}

\textbackslash{}begin\{figure\}
\includegraphics{report-test-odt_files/figure-latex/spp_dist_std1-1}
\textbackslash{}caption{[}Diameter distribution for each species that
makes up at least 8\% of the total stand stocking{]}\{Diameter
distribution for each species that makes up at least 8\% of the total
stand stocking. Basal areas are represented by the areas under each
curve.\}\label{fig:spp_dist_std1} \textbackslash{}end\{figure\}

\subsection{Long-term vision}\label{long-term-vision}

Eventually, we aim to develop a forest with a diverse mix of high value
hardwood species and white pine. Management will focus on growing high
quality veneer and sawtimber, and trees will be grown to large sizes to
maximize their value growth. Most of the trees present today are not
high quality, and---given their large size---have reached financial
maturity. In the near term, the focus should be on effectively
regenerating desirable trees to replace them. The minority of pines that
are of investment grade are not yet financially mature (higher quality
trees are financially mature at larger sizes) and should be retained
through the regeneration period and into the subsequent rotation.

\subsection{Long-term management
system}\label{long-term-management-system}

\textbf{Even-aged, crop tree management}\\
\textbf{References:} Perkey, A.W., B.L. Wilkins, and H.C. Smith. 1994.
Crop Tree Management in Eastern Hardwoods. USDA For. Serv. NA-TP-19-93.;
Leak, W.B., M.Yamasaki, and R. Holleran. 2014. Silvicultural Guide for
Northern Hardwoods in the Northeast. USDA For. Serv. Gen.~Tech.
Rep.~NRS-132.

The stand will be regenerated using a shelterwood system, which will
make best use of the few valuable pines that are present while providing
good conditions for the establishment of new pines and hardwoods. The
subsequent rotation will be mostly even-aged, but investment-grade stems
from this rotation will be reserved until they reach maturity in several
decades. Once they have developed clear lower boles (in three or four
decades), young trees will be tended regularly using crop tree
management, in which the best trees are identified and released from
competition to speed their growth. This strategy will produce the
highest quality timber and make for an appealing atmosphere with large,
healthy trees. The stand should be grown to approximately 120 years
before it is regenerated again.

\subsection{Silvicultural
prescription}\label{silvicultural-prescription}

\textbf{Shelterwood establishment cut}\\
\textbf{Year:} 2021\\
\textbf{References:} Lancaster, K.F. and W.B. Leak. 1978. A
silvicultural guide for white pine the Northeast. USDA For. Serv.
Gen.~Tech. Rep.~NE-41. 13 p.; Leak, W.B., M.Yamasaki, and R. Holleran.
2014. Silvicultural Guide for Northern Hardwoods in the Northeast. USDA
For. Serv. Gen.~Tech. Rep.~NRS-132. Pp 29-30.

A shelterwood establishment cut should be used to regenerate a mix of
white pine, sugar maple, yellow birch, and other valuable hardwoods. The
overstory basal area should be reduced to approximately
60ft\textsuperscript{2}/ac, which meets guidelines for both white pine
and northern hardwood regeneration. All inventory grade trees should be
retained, along with enough of the highest quality non-inventory grade
pine AGS that are available to achieve the desired stocking. In
conjunction with the harvest, invasive honeysuckle plants should be
removed mechanically or with herbicide to prevent them inhibiting pine
and hardwood regeneration. The treatment should ideally take place when
the ground is bare so the soil can be scarified to encourage pine
regeneration; however it will need to be timed to coincide with other
work on the property, which must happen in winter. Perhaps this work can
be done in the fall at the beginning of the logging operation, or in the
spring at the end.

\subparagraph{pagebreak}\label{pagebreak-2}

\section{Stand 2}\label{stand-2}

\textbf{Mixedwood}\\
\textbf{46.92 legal acres -- 52.37 measured acres}

\subsection{Stand conditions}\label{stand-conditions-1}

Area 2 is almost all underlain by Buckland soils, and tends to be better
drained and somewhat more productive than other areas. Like Area 1, it
slopes gently to the east. It is located behind Area 1 and the pond, and
extends to Jug Brook in the south. Most of the property's recreational
trails are located in Area 2.

A mix of hardwoods and softwoods is present in the area, but hardwoods
predominate. A tending operation in 2000 and a salvage operation in 2006
removed much softwood and pushed the composition toward hardwoods. The
2006 logging removed blowdown in the wetter eastern part of the stand
behind the pond (where the only Cabot soils are located), and left
overstory stocking in that area as low as 60 or 70 square feet, compared
to stocking above 100 square feet on most of the rest of the stand.
Older logging jobs appear to have preferentially removed higher quality
timber and ¾ of the growing stock is now incapable of growing high grade
sawlogs or veneer, or is poorly formed with very limited timber volume.
Over about half of the stand, the stocking of acceptable growing stock
is below c-line and is incapable of fully occupying the stand again in
this rotation.

Advance regeneration is well established across the stand, and
constitutes a second cohort in most places. It was established after the
recent logging operations, so it ranges from 15 to 20 years old. Areas
that had advance fir regeneration before the logging now host more than
1000 fir saplings to the acre (especially closer to Jug Brook), while
the salvaged area in the east has an abundance of sugar maple, fir,
cherry and ash saplings. Where the logging was lighter, spruce saplings
are present in lower numbers, but still generally sufficient to fully
stock a new stand. Honeysuckle is scattered very lightly throughout,
though it will not be able to compete with the well established tree
regeneration and is not a great concern.

\subsection{Quantitative stand data}\label{quantitative-stand-data-1}

\textbf{Site Class: } II (soils mapping \& field assessment)\\
\textbf{Access Distance: } Less than one mile\\
\textbf{Cruise Data: } 10 points, 10 BAF, Sep. 2018\\
\textbf{Age Class Structure: } Even-aged (80 or 90 years)\\
\textbf{Volume/ac: } 0 MBF veneer, 4.3 MBF sawtimber, 1.5 MBF tie logs,
\& 14 cds pulp

\textbf{Size Class Structure (\%BA): } 6-10'': 44\% --- 10-16'': 32\%
--- 16-22'': 14\% --- 22+'': 10\%

line

\begin{longtable}[]{@{}lrrr@{}}
\caption{Measures of stocking for all live trees in the main canopy
(total), acceptable growing stock, and inventory-grade growing stock
(which is a subset of acceptable growing stock).}\tabularnewline
\toprule
& Total & Acceptable growing stock & Inventory-grade\tabularnewline
\midrule
\endfirsthead
\toprule
& Total & Acceptable growing stock & Inventory-grade\tabularnewline
\midrule
\endhead
Basal area (sqft/ac) & 102 & 76 & 27\tabularnewline
QSD (in) & 12 & 12 & 13\tabularnewline
Stems/ac & 154 & 116 & 44\tabularnewline
\bottomrule
\end{longtable}

\begin{longtable}[]{@{}lr@{}}
\caption{Species present and their percentage of total stand basal
area.}\tabularnewline
\toprule
Species & \% BA\tabularnewline
\midrule
\endfirsthead
\toprule
Species & \% BA\tabularnewline
\midrule
\endhead
soft maple & 30\tabularnewline
hard maple & 25\tabularnewline
ash & 11\tabularnewline
hemlock & 7\tabularnewline
spruce & 7\tabularnewline
fir & 6\tabularnewline
white pine & 5\tabularnewline
yellow birch & 5\tabularnewline
paper birch & 3\tabularnewline
aspen & 1\tabularnewline
black cherry & 1\tabularnewline
\bottomrule
\end{longtable}

\textbackslash{}begin\{figure\}
\includegraphics{report-test-odt_files/figure-latex/spp_dist_std2-1}
\textbackslash{}caption{[}Diameter distribution for each species that
makes up at least 8\% of the total stand stocking{]}\{Diameter
distribution for each species that makes up at least 8\% of the total
stand stocking. Basal areas are represented by the areas under each
curve.\}\label{fig:spp_dist_std2} \textbackslash{}end\{figure\}

\subsection{Long-term vision}\label{long-term-vision-1}

Like Area 1, this stand contains little valuable growing stock and
should be regenerated so that new trees will be available to replace the
old. Unlike Area 1, the regeneration process was already started in 2000
(when logging triggered the establishment of many young stems) and we
will continue the process step-wise to develop an uneven-aged forest,
which will better suit the landowners. Eventually, we hope to establish
a diverse, vibrant forest, with trees of many ages living together, that
will support many wildlife species and provide regular logging revenue.
A focus on quality will encourage trees with big crowns and straight
trunks that will be grown to large sizes.

\subsection{Long-term management
system}\label{long-term-management-system-1}

\textbf{Group selection}\\
\textbf{References:} Nyland, R.D. 2002. Silviculture: Concepts and
Applications. 2nd edition. Waveland Press Inc., Long Grove, IL. p.~248;
Leak, W.B., M.Yamasaki, and R. Holleran. 2014. Silvicultural Guide for
Northern Hardwoods in the Northeast. USDA For. Serv. Gen.~Tech.
Rep.~NRS-132.

The vision above will be best met using a group-selection system, in
which ½ to 2 acre areas will host groups of trees of the same age. By
cutting about 1/6\textsuperscript{th} to 1/8\textsuperscript{th} of the
stand in these groups every 15 or 20 years, new trees will be
continuously regenerated and groups of many ages will always be present.
Once a group is about 120 years old, its trees will be harvested and the
area will be regenerated. This system will allow enough light into
regenerating areas to recruit shade intermediate species like yellow
birch, white ash, and black cherry in addition to the more shade
tolerant sugar maple and spruce. Because there is presently so little
quality growing stock to work with, a greater area of the stand should
be regenerated now to kick start the system. That will provide a flush
of younger trees to make up for the poor vigor trees that are standing,
and which will probably not live through the whole transition period.
The extra regeneration will eventually be cut prematurely to achieve a
balanced structure overall. Establishing this system will realistically
take 75 years or more, but every entry will increase the value and
diversity of the stand, and the current landowner and foresters will get
to preside over a process of continual improvement.

\subsection{Silvicultural
prescription}\label{silvicultural-prescription-1}

\textbf{Group selection -- uneven-aged conversion cut}\\
\textbf{Year:} 2021\\
\textbf{References:} Nyland, R.D. 2003. \emph{Even- to Uneven-aged: the
Challenges of Conversion}. Forest Ecology and Management: 172. pp
291-300.; Leak, W.B., M.Yamasaki, and R. Holleran. 2014. Silvicultural
Guide for Northern Hardwoods in the Northeast. USDA For. Serv.
Gen.~Tech. Rep.~NRS-132.

The primary objective of this treatment is to begin to regulate the
structure of this degraded stand by releasing avance regeneration in
areas (or regenerating where advance regeneration is inadequate), while
leaving an intact canopy over most of the stand area. Species targeted
for regeneration (where it is not already present) are sugar maple, red
spruce, yellow birch, ash, and black cherry. Group openings should be
created that are at least ½ acre and do not exceed 2 acres, and the
total area of openings should not exceed 1/3 of the stand area
(aproximately 17 acres). A wide range of group size, shape, and
orientation arrangements should be created to accommodate the
establishment and recruitment of regeneration of varying degrees of
shade tolerance; and groups may run together, as long as their interiors
remain within about one tree-height of the residual forest (to prevent
the regeneration of large numbers of shade-intolerants). Openings should
be distributed throughout the stand and represent a variety of site
conditions.

No tending should be done in the matrix between openings, to prevent
undue damage to residual trees. Stocking is currently below b-line and
residuals will continue growing well without any attention now. We
expect the species composition of the residual stand to largely resemble
the current composition.

Openings and skid trails should not be located within 50 feet of stream
banks (except around stream crossings), to prevent stream sedementation,
protect water quality, and maintain canopy shade for aquatic organisms.
Standing dead trees should also be retained within openings so they can
be used by cavity dwelling animals and as perches.

\subparagraph{pagebreak}\label{pagebreak-3}

\section{Stand 3}\label{stand-3}

\textbf{Mixedwood}\\
\textbf{101.56 legal acres -- 113.35 measured acres}

\subsection{Stand conditions}\label{stand-conditions-2}

Area 3 comprises most of the land south of Jug Brook, and a stream
crossing will need to be developed for any vehicular access unless the
neighbors allow access from McCarty Road in the southeast. Cabot and
Buckland soils are intermixed on the stand, with Cabot soils generally
found further south; and the site class is variable. Areas of more
poorly drained soils can be identified by shorter canopy heights,
indicating lower productivity. In the previous management plan, Area 3
included the land now mapped as Area 4. We have decided to differentiate
them in this plan because Area 4 is compositionally distinct and will be
treated differently from Area 3.

Some of the area was thinned in 2000 along with the logging in Areas 1
and 2, but much of the stand was left alone and the stocking is more
uniform and higher than that of Area 2; though it still hovers around
b-line on the mixedwood stocking chart. Older logging operations
apparently degraded the timber quality, as the stocking of acceptable
growing stock is right around c-line and only half of that is
inventory-grade.

Advance regeneration is fairly well established, and is composed of a
mix of hardwood and softwood species, including sugar maple, red maple,
ash, and fir. In some places, that regeneration was the result of
windthrow in the overstory. Windthrow continues to be a concern in
wetter spots.

\subsection{Quantitative stand data}\label{quantitative-stand-data-2}

\textbf{Site Class: } II \& III (soils mapping \& field assessment)\\
\textbf{Access Distance: } Less than one mile\\
\textbf{Cruise Data: } 21 points, 10 BAF, Sep. 2018\\
\textbf{Age Class Structure: } Even-aged (80 or 90 years)\\
\textbf{Volume/ac: } 0.2 MBF veneer, 4.6 MBF sawtimber, 1.8 MBF tie
logs, \& 15 cds pulp

\textbf{Size Class Structure (\%BA): } 6-10'': 46\% --- 10-16'': 39\%
--- 16-22'': 10\% --- 22+'': 5\%

line

\begin{longtable}[]{@{}lrrr@{}}
\caption{Measures of stocking for all live trees in the main canopy
(total), acceptable growing stock, and inventory-grade growing stock
(which is a subset of acceptable growing stock).}\tabularnewline
\toprule
& Total & Acceptable growing stock & Inventory-grade\tabularnewline
\midrule
\endfirsthead
\toprule
& Total & Acceptable growing stock & Inventory-grade\tabularnewline
\midrule
\endhead
Basal area (sqft/ac) & 115 & 89 & 43\tabularnewline
QSD (in) & 11 & 11 & 11\tabularnewline
Stems/ac & 215 & 175 & 93\tabularnewline
\bottomrule
\end{longtable}

\begin{longtable}[]{@{}lr@{}}
\caption{Species present and their percentage of total stand basal
area.}\tabularnewline
\toprule
Species & \% BA\tabularnewline
\midrule
\endfirsthead
\toprule
Species & \% BA\tabularnewline
\midrule
\endhead
hard maple & 22\tabularnewline
yellow birch & 21\tabularnewline
hemlock & 18\tabularnewline
ash & 14\tabularnewline
soft maple & 10\tabularnewline
fir & 7\tabularnewline
aspen & 4\tabularnewline
paper birch & 3\tabularnewline
spruce & 2\tabularnewline
other hardwood & 0\tabularnewline
\bottomrule
\end{longtable}

\textbackslash{}begin\{figure\}
\includegraphics{report-test-odt_files/figure-latex/spp_dist_std3-1}
\textbackslash{}caption{[}Diameter distribution for each species that
makes up at least 8\% of the total stand stocking{]}\{Diameter
distribution for each species that makes up at least 8\% of the total
stand stocking. Basal areas are represented by the areas under each
curve.\}\label{fig:spp_dist_std3} \textbackslash{}end\{figure\}

\subsection{Long-term vision}\label{long-term-vision-2}

An uneven-aged forest will best meet the landowners' objectives in this
area by providing regular logging revenue while maintaining aesthetic
appeal and providing varied interior habitat for a variety of animals.
It and Area 2 will eventually look very simlar: diverse, vibrant
forests, with vigorous, valuable trees of many ages. Trees will have
straight, branch-free lower trunks and large, healthy crowns. Different
areas within the stand will host trees of different ages, so people (and
animals) walking through will encounter a variety of conditions; from
dense, private-feeling spots dominated by young trees, to areas of
impressive, tall trees that are easy to walk through and provide long
lines of sight.

\subsection{Long-term management
system}\label{long-term-management-system-2}

\textbf{Group selection}\\
\textbf{References:} Nyland, R.D. 2002. Silviculture: Concepts and
Applications. 2nd edition. Waveland Press Inc., Long Grove, IL. p.~248;
Leak, W.B., M.Yamasaki, and R. Holleran. 2014. Silvicultural Guide for
Northern Hardwoods in the Northeast. USDA For. Serv. Gen.~Tech.
Rep.~NRS-132.

Long-term, the management system in Area 3 will be identical to that in
Area 2, and the two stands will merge. One sixth to
1/8\textsuperscript{th} of the stand will be regenrated in group
openings every 15 or 20 years, and groups will be grown to approximately
120 years before being harvested. Also, as in Area 2, a greater area of
the stand should be regenerated now because of the lack of quality
growing stock and the need to shorten the length of time residual trees
will need to be retained. There is more quality growing stock in Area 3
than in Area 2, however, and its treatment can be somewhat lighter.

\subsection{Silvicultural
prescription}\label{silvicultural-prescription-2}

\textbf{Group selection -- uneven-aged conversion cut}\\
\textbf{Year:} 2021\\
\textbf{References:} Nyland, R.D. 2003. \emph{Even- to Uneven-aged: the
Challenges of Conversion}. Forest Ecology and Management: 172. pp
291-300.; Leak, W.B., M.Yamasaki, and R. Holleran. 2014. Silvicultural
Guide for Northern Hardwoods in the Northeast. USDA For. Serv.
Gen.~Tech. Rep.~NRS-132.

The primary objective of this treatment is to begin to regulate the
structure of this degraded stand by regenerating a diverse cohort of
trees, while leaving an intact canopy over most of the stand area.
Species targeted for regeneration are sugar maple, red spruce, yellow
birch, ash, and black cherry. Group openings should be created that are
at least ½ acre and do not exceed 2 acres, and the total area of
openings should not exceed 1/4 of the stand area (aproximately 28
acres). A wide range of group size, shape, and orientation arrangements
should be created to accommodate the establishment and recruitment of
regeneration of varying degrees of shade tolerance; and groups may run
together, as long as their interiors remain within about one tree-height
of the residual forest (to prevent the regeneration of large numbers of
shade-intolerants). Openings should be distributed throughout the stand
and represent a variety of site conditions.

No tending should be done in the matrix between openings, to prevent
undue damage to residual trees. Stocking is currently at b-line and
residuals will continue growing well without any attention now. We
expect the species composition of the residual stand to largely resemble
the current composition.

Openings and skid trails should not be located within 50 feet of stream
banks (except around stream crossings), to prevent stream sedementation,
protect water quality, and maintain canopy shade for aquatic organisms.
Standing dead trees should also be retained within openings so they can
be used by cavity dwelling animals and as perches.

\subparagraph{pagebreak}\label{pagebreak-4}

\section{Stand 4}\label{stand-4}

\textbf{White Pine}\\
\textbf{13.12 legal acres -- 14.64 measured acres}

\subsection{Stand conditions}\label{stand-conditions-3}

This stand surrounds an abandoned house site that is now marked by house
and barn cellar holes and an old spring. The homestead was probably
originally accessed from McCarty Road ---- a dead end road that has been
partially abandoned and that connects to Jug Brook Road further east.
Area 4 made up the homestead's more intensively used, plowed
agricultural land; and the resulting soil compaction favored pine
regeneration when the field was eventually abandoned. It is also
possible that the field was planted to pine when grazing and mowing
stopped.

The 2000 logging operation did not affect the area, though it has been
thinned in the more distant past. The stocking of acceptable growing
stock remains above b-line, but past logging operations, weevil damage,
and some blister rust have left only 40 square feet of inventory-grade
growing stock. The total stocking is midway between a- and b-line on
Leak and Lampson's stocking guide. While the site is mostly level and
dry, a small area in the northeast has wet soils that retard tree
growth.

Some hard maple, ash, and black cherry advance regeneration has become
established under the pines throughout, and a few invasive honeysuckles
were found in the wet area.

\subsection{Quantitative stand data}\label{quantitative-stand-data-3}

\textbf{Site Class: } II (soils mapping \& field assessment)\\
\textbf{Access Distance: } Less than one mile\\
\textbf{Cruise Data: } 4 points, 10 BAF, Sep. 2018\\
\textbf{Age Class Structure: } Even-aged (\textasciitilde{}100 years?)\\
\textbf{Volume/ac: } 0 MBF veneer, 17.4 MBF sawtimber, 4.1 MBF tie logs,
\& 21 cds pulp

\textbf{Size Class Structure (\%BA): } 6-10'': 19\% --- 10-16'': 34\%
--- 16-22'': 23\% --- 22+'': 24\%

line

\begin{longtable}[]{@{}lrrr@{}}
\caption{Measures of stocking for all live trees in the main canopy
(total), acceptable growing stock, and inventory-grade growing stock
(which is a subset of acceptable growing stock).}\tabularnewline
\toprule
& Total & Acceptable growing stock & Inventory-grade\tabularnewline
\midrule
\endfirsthead
\toprule
& Total & Acceptable growing stock & Inventory-grade\tabularnewline
\midrule
\endhead
Basal area (sqft/ac) & 208 & 145 & 40\tabularnewline
QSD (in) & 14 & 15 & 17\tabularnewline
Stems/ac & 206 & 124 & 24\tabularnewline
\bottomrule
\end{longtable}

\begin{longtable}[]{@{}lr@{}}
\caption{Species present and their percentage of total stand basal
area.}\tabularnewline
\toprule
Species & \% BA\tabularnewline
\midrule
\endfirsthead
\toprule
Species & \% BA\tabularnewline
\midrule
\endhead
white pine & 58\tabularnewline
soft maple & 24\tabularnewline
fir & 10\tabularnewline
black cherry & 4\tabularnewline
hard maple & 4\tabularnewline
tamarack & 1\tabularnewline
\bottomrule
\end{longtable}

\textbackslash{}begin\{figure\}
\includegraphics{report-test-odt_files/figure-latex/spp_dist_std4-1}
\textbackslash{}caption{[}Diameter distribution for each species that
makes up at least 8\% of the total stand stocking{]}\{Diameter
distribution for each species that makes up at least 8\% of the total
stand stocking. Basal areas are represented by the areas under each
curve.\}\label{fig:spp_dist_std4} \textbackslash{}end\{figure\}

\subsection{Long-term vision}\label{long-term-vision-3}

This stand will be kept even-aged and grown to the end of its rotation
to take advantage of the growing stock that is not yet mature. It will
also provide a component of large trees and softwood cover for wildlife
species that rely on those characteristics. While the stand looks to be
over 100 now, it should be grown another 20 years or so before a
regeneration sequence is intiated. A vision for the subsequent rotation
can be developed at that time.

\subsection{Long-term management
system}\label{long-term-management-system-3}

\textbf{Even-aged system}\\
\textbf{References:} Lancaster, K.F. and W.B. Leak. 1978. A
silvicultural guide for white pine the Northeast. USDA For. Serv.
Gen.~Tech. Rep.~NE-41. 13 p.

Thinnings in this stand should generally occur every 15 to 20 years
until the rotation is completed at age 120 or 130. The next entry will
be the last thinning before regeneration begins.

\subsection{Silvicultural
prescription}\label{silvicultural-prescription-3}

\textbf{Crown thinning}\\
\textbf{Year:} 2021\\
\textbf{References:} Lancaster, K.F. and W.B. Leak. 1978. A
silvicultural guide for white pine the Northeast. USDA For. Serv.
Gen.~Tech. Rep.~NE-41. 13 p.; Leak, W.B. and N.I. Lamson. 1999. Revised
white pine stocking guide for managed stands. USDA For. Serv. Tech. Pap.
NA-TP-01-99. 2 p.

Stocking should be reduced to about 110 ft\textsuperscript{2}/ac
(managed b-line on Leak and Lamson's stocking chart) by removing
poorer-quality codominant trees and unaccptable dominant trees. The
highest quality pine stems should be retained and favored so they have
space to grow. The foundations should be clearly marked during the
operation so they aren't damaged.

\subparagraph{pagebreak}\label{pagebreak-5}

\section{Glossary}\label{glossary}

\textbf{Inventory-grade growing stock:}



\end{document}
